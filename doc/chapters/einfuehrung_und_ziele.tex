%! Author = nico
%! Date = 20.05.23

\section{Einführung und Ziele}
\label{sec:einfuehrung_und_ziele}
In folgendem Kapitel wird die Aufgabenstellung und die Ziele der Arbeit erläutert.
Es wird auf die Motivation eingegangen und die Anforderungen werden definiert.

\subsection{Aufgabenstellung}
\label{subsec:aufgabenstellung}
Im Rahmen der Modularbeit wird eine Web-Applikation spezifiziert und entwickelt.
Diese Modularbeit widerspiegelt die verschiedenen Architekturstile und -konzepte,
die während des Moduls behandelt werden.
Die Lösung basiert auf Jakarta EE, wird  um die Konzepte von Spring Boot erweitert und
mit in Micronaut erstelle REST Services ergänzt.
Die verschiedenen Services oder Komponenten werden letztlich als
eigenständige Microservices als Container in Docker und Kubernetes verteilt.

Die Modularbeit wird auf eine Arbeit aus dem Modul PA5 aufgebaut. Im Rahmen dieser Arbeit
wurde ein Inventarverwaltungstool für Brauereien entwickelt. Nun soll ein Webshop zwecks
dem Verkauf dieser gebrauten Biere entwickelt werden. Die verschiedenen Komponenten der 
Applikation werden als Microservices aufgebaut. Auf dem Webshop soll eine Produktübersicht 
vorhanden sein. Die Produkte bieten eine ausführliche Beschreibung inklusive Zutaten und Bildern. 
Sie können in den Warenkorb gelegt und anschliessend bestellt werden. Die Verwaltung der 
verschiedenen Objekte soll sich auf ein Minimum beschränken.

\subsection{Funktionale Anforderungen}
\label{subsec:funktionaleanforderungen}
Es soll ein einfacher Webshop entwickelt werden, der die folgenden Funktionen bietet:

\begin{itemize}
    \item \textbf{Produktübersicht:} Auf dem Webshop soll eine Produktübersicht vorhanden sein.
    \item \textbf{Produktdetails:} Die Produkte bieten eine ausführliche Beschreibung inklusive Zutaten, Bildern, Preis und Verfügbarkeit.
    \item \textbf{Warenkorb:} Die Produkte können in den Warenkorb gelegt werden.
    \item \textbf{Bestellung:} Die Produkte können bestellt werden.
\end{itemize}

\subsection{Qualitätsziele an die Architektur}
\label{subsec:qualitaetsziele}
Die folgenden Qualitätsziele sollen bei der Entwicklung der Architektur berücksichtigt werden:

\begin{itemize}
    \item \textbf{Performance:} Die Performance der Anwendung soll so gut wie möglich sein.
    Es soll eine möglichst hohe Anzahl an Anfragen pro Sekunde verarbeitet werden können.
    \item \textbf{Skalierbarkeit:} Die Anwendung soll skalierbar sein. Es soll möglich sein,
    mehrere Instanzen der Anwendung parallel zu betreiben.
    \item \textbf{Erweiterbarkeit:} Die Anwendung soll möglichst einfach erweiterbar sein. 
    \item \textbf{Wartbarkeit:} Die Anwendung soll möglichst einfach wartbar sein.
    Fehler sollen leicht auffindbar und behebbar sein.
    \item \textbf{Testbarkeit:} Die Anwendung soll möglichst einfach testbar sein.
\end{itemize}

\subsection{Stakeholder}
\label{subsec:stakeholder}

\begin{table}[H]
    \centering
    \begin{tabular}{|l|l|l|}
        \hline
        \textbf{Rolle}              & \textbf{Kontakt}  & \textbf{Erwartungshaltung} \\ \hline

        Entwicklerin                & Miriam Streit     & \makecell{Umsetzung der Applikation \\ gemäss Aufgabenstellung} \\ \hline
        Entwickler / Product Owner  & Nico Berchtold    & \makecell{Umsetzung der Applikation \\ gemäss Aufgabenstellung} \\ \hline
        Dozent                      & Daniel Senften    & \makecell{Umsetzung der Applikation \\ gemäss Aufgabenstellung} \\ \hline
        Benutzer                    & -                 & \makecell{Einfache Bedienung \\ der Applikation} \\ \hline
    \end{tabular}
    \caption{Stakeholder}
    \label{tab:stakeholder}
\end{table}