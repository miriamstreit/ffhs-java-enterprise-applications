%! Author = nico
%! Date = 20.05.23

\section{Kontextabgrenzung}
\label{sec:kontextabgrenzung}

\subsection{Systemkontext}
\label{subsec:systemkontext}
Zum aktuellen Zeitpunkt verfügt das System keine Schnittstellen nach aussen. Es handelt sich um eine eigenständige Einheit, die in sich abgeschlossen ist und keine Kommunikation oder Interaktion andern Systemen hat. Der gesamte Funktionsumfang vom Inventar des Shops, der Bestellung bin hin zur Verwaltung der Kunden wird im System abgebildet.
Zu einem späteren Zeitpunkt ist es jedoch denkbar, dass das System mit dem BrewBuddy-System kommuniziert und die Beer-Sorten von dort bezieht. Ebenso ist es naheliegend, dass das System in Zukunft mit einem Payment-Provider kommuniziert, um die Bezahlung der Bestellungen abzuwickeln.

\subsection{Fachlicher Kontext}
\label{subsec:fachlicherkontext}
Das Fachwissen der Applikation, also das Inventar für die verschiedenen Biersorten muss manuell in der Datenbank eingetragen werden. Kunden sowie Bestellungen werden vom System automatisch abgewickelt.