%! Author = Miriam Streit
%! Date = 01.07.23

\section{Installation}
\label{sec:installation}

Sämtliche Dokumentation zur Installation befindet sich in der Datei README.md auf der obersten Ebene des Repository. Gemäss dieser Anleitung kann die Applikation entweder auf einem
lokalen Minikube Kubernetes Cluster oder lokal mit Docker-Compose betrieben werden. 

Wir hatten immer wieder sehr viel Mühe, die JakartaEE-Applikation (den Product-Service) überall zum Laufen zu bringen. Bei Miriam (MacOS ohne M1) funktionierte alles reibungslos,
bei Nico (MacOS mit M1) funktionierte der Product-Service nur lokal im IDE. Im Container-Betrieb funktionierte der Service bei Nico überhaupt nicht. Da die Produkte zentral für die gesamte
Applikation sind, wurde der Product-Service zusätzlich mit Spring Boot umgesetzt (product-service-boot). Dieser Service funktioniert auf allen Plattformen und kann als Alternative
verwendet werden, damit die Funktionalität trotzdem getestet werden kann. Genaueres ist ebenfalls in der Installationsanleitung zu finden.

Die erfolgreich deployte Applikation (unabhängig davon, ob mit Docker-Compose oder Kubernetes) sollte dann intuitiv zu bedienen sein. Sie bietet eine Produktübersicht, einen Warenkorb
und ein Bestellformular.